\chapter{Εισαγωγή} 
\InitialCharacter{Α}πό την εποχή που γεννήθηκε η ιδέα του Παγκόσμιου Ιστού ως ένα διεθνές σύστημα διακίνησης πληροφοριών μέχρι σήμερα,
έχουμε διανύσει τρεις δεκαετίες απροσδόκητης τεχνολογικής έκρηξης \cite{Web89}.
Η αξιοποίηση του διαδικτύου πλέον δεν αφορά μόνο ακαδημαϊκούς υπολογιστές για ερευνητικούς σκοπούς,
αλλά αποτελεί αναπόσπαστο κομμάτι της καθημερινότητας της πλειονότητας των ανθρώπων παγκοσμίως.
Με την επικράτηση των προσωπικών υπολογιστών και στη συνέχεια των έξυπνων κινητών (\en{smartphones}),
η ψηφιακή διακίνηση πληροφορίας διευρύνεται και ενισχύεται ασταμάτητα.
Όσο για το άμεσο μέλλον, βέβαιες πρέπει να θεωρούνται η σταδιακή μεταβίβαση στο Διαδίκτυο των Πραγμάτων (\en{IoT}) και η καθολικότητα των έξυπνων συσκευών.

Για την αξιοποίηση του τεράστιου όγκου της πληροφορίας που διακινείται μέσω του Παγκόσμιου Ιστού χρησιμοποιούνται διεπαφές προγραμματισμού.
Σκοπός αυτού του μηχανισμού επικοινωνίας είναι η αποδοχή αιτημάτων από άλλα προγράμματα του διαδικτύου και η επεξεργασία τους,
με στόχο τη μεταφορά κατάλληλων μηνυμάτων σε υπολογιστές και την επιστροφή των αντιστοίχων αποτελεσμάτων στους χρήστες \cite{date_relational_1975}.

Η αδιάλειπτη ανάπτυξη του Παγκόσμιου Ιστού καθιστά περισσότερο αναγκαίο από ποτέ τον αποτελεσματικό έλεγχο των διεπαφών.
Οι προγραμματιστές οφείλουν να είναι βέβαιοι πως οι εφαρμογές που αναπτύσσουν παράγουν πάντα τα επιθυμητά αποτελέσματα, 
με την κατάλληλη επεξεργασία κάθε φορά της μεταδιδόμενης πληροφορίας.
Μόνο έτσι μπορεί να εδραιωθεί η αξιοπιστία των διαδικτυακών εφαρμογών,
τόσο για τους προγραμματιστές που καλούνται να αναπτύξουν νέα προγράμματα που θα αλληλεπιδρούν με τα υπάρχοντα,
όσο και για τους απλούς χρήστες που απαιτούν ασφάλεια και σιγουριά κατά την ψηφιακή πλοήγησή τους.

\section{Αντικείμενο διπλωματικής εργασίας}
Αντικείμενο της διπλωματικής εργασίας είναι η ανάπτυξη ενός εργαλείου για τον αυτόματο έλεγχο διεπαφών προγραμματισμού τύπου \en{REST} (\en{RESTful APIs}).
Με το εργαλείο αυτό οι προγραμματιστές θα μπορούν να ελέγχουν πιο αποτελεσματικά τον κώδικά τους, καθώς θα τους παρέχει έτοιμα σενάρια ελέγχου.
Τα σενάρια αυτά θα είναι παραμετροποιήσιμα, θα καλύπτουν κατά το δυνατόν πληρέστερα τις δυνατότητες μιας διεπαφής και η δημιουργία τους θα γίνεται αυτόματα.
Το μόνο που απαιτείται από τον χρήστη είναι η περιγραφή της διεπαφής σε κατάλληλη μορφή.

\section{Συνεισφορά διπλωματικής εργασίας}
Σκοπός της διπλωματικής εργασίας είναι η προσφορά στην κοινότητα προγραμματιστών \en{Java} ενός εργαλείου
για τον αποδοτικό έλεγχο προγραμματιστικών διεπαφών τύπου \en{REST}.

Το εργαλείο αυτό σχεδιάστηκε με γνώμονα την βέλτιστη δυνατή εμπειρία προγραμματισμού.
Χρησιμοποιώντας το, οι προγραμματιστές είναι σε θέση να παρέχουν εγγυήσεις ποιότητας για τα προϊόντα τους,
μιας και η αυτόματη παραγωγή των σεναρίων ελέγχου συνεπάγεται την έλλειψη ανθρωπίνων λαθών, τόσο συντακτικών όσο και λογικών.

Ταυτόχρονα επιτυγχάνεται αύξηση της παραγωγικότητας, 
αφού αρκεί η κατάλληλη περιγραφή των προδιαγραφών της προγραμματιστικής διεπαφής τύπου \en{REST} για να παραχθούν αμέσως όλα τα δυνατά σενάρια ελέγχου που καλύπτονται από το πεδίο εφαρμογής της εργασίας.

\section{Οργάνωση του τόμου}
Η εργασία αυτή είναι οργανωμένη σε επτά κεφάλαια, συμπεριλαμβανομένου του παρόντος και πρώτου που περιλαμβάνει το αντικείμενο της εργασίας,
τη συνεισφορά της στην προγραμματιστική κοινότητα και την οργάνωσή της.
Στο Κεφάλαιο 2 δίνεται μια εισαγωγή στον αυτόματο έλεγχο προγραμματιστικών διεπαφών διαδικτύου,
όπου αναπτύσσονται οι έννοιες των \en{Web APIs} και του ελέγχου λογισμικού. 
Στο Κεφάλαιο 3 περιγράφεται ο σχεδιασμός της εργασίας, ενώ στο Κεφάλαιο 4 παρουσιάζεται αναλυτικά η υλοποίησή της. 
Στη συνέχεια, στο Κεφάλαιο 5 παρατίθενται εφαρμογές σε δύο διαδικτυακά \en{RESTful APIs} που επικυρώνουν την ορθή λειτουργία της εργασίας.
Στο Κεφάλαιο 6 γίνεται αναφορά σε συναφή προγραμματιστικά εργαλεία και ακαδημαϊκές μελέτες.
Τέλος, στο Κεφάλαιο 7 δίνονται τα συμπεράσματα της εργασίας και σχολιάζονται ορισμένες μελλοντικές επεκτάσεις της.