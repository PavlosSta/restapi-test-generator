\chapter{Σχετικές Εργασίες}
\InitialCharacter{Η} μελέτη των προγραμματιστικών διεπαφών τύπου \en{REST} και του ελέγχου τους,
παρότι είναι ένας τομέας που αξιοποιείται σημαντικά σε επιχειρησιακό επίπεδο,
παρουσιάζει και μεγάλο ερευνητικό ενδιαφέρον σήμερα.

Οι προγραμματιστικές διεπαφές τύπου \en{REST} μελετώνται παράλληλα με τις σύγχρονες τεχνολογικές τάσεις,
όπως είναι για παράδειγμα η ανάπτυξη της νέας γενιάς ασύρματων δικτύων \en{5G} \cite{mayer2018restful}.
Η νέα γενιά αυτή αναμένεται να επιφέρει ραγδαίες εξελίξεις ως προς τη μετάβαση στο διαδίκτυο των πραγμάτων (\en{IoT}) \cite{wang2018iot},
μία κατάσταση στην οποία μηχανές και συσκευές καθημερινής χρήσης αλληλεπιδρούν μεταξύ τους 
μέσω ενος διεθνούς δικτύου \cite{gubbi2013internet}. 

Με την έκρηξη της μηχανικής μάθησης και των νευρωνικών δικτύων,
έχουν υπάρξει προσπάθειες αξιοποίησης της τεχνητής νοημοσύνης στον κλάδο των προγραμματιστικών διεπαφών.
Αυτές περιλαμβάνουν 
τόσο τη σχεδίαση νέων μορφών διεπαφών, προσαρμοσμένων στις απαιτήσεις της μηχανικής μάθησης \cite{garcia2019deepaas}\cite{howard2020fastai}
και την σύγκριση των δημοφιλέστερων \cite{kubany2020comparison},
όσο και τον έλεγχο των διεπαφών με την αυτοματοποιημένη παραγωγή μεγάλου όγκου σεναρίων ελέγχου 
που προκύπτουν από την ανάλυση χιλιάδων άλλων προγραμμάτων \cite{martin2020ai}.

Στο παρόν κεφάλαιο περιγράφονται τα παρακάτω πιο δημοφιλή εργαλεία που χρησιμοποιούνται σήμερα 
για την μοντελοποίηση και τον έλεγχο προγραμματιστικών διεπαφών τύπου \en{REST}:

\begin{itemize}
    \item \en{OpenAPI Specification} \cite{noauthor_oaiopenapispecification_2021}
    \item \en{Swagger Inspector} \cite{swagger}
    \item \en{Postman} \cite{postman}
\end{itemize}

\section{\en{OpenAPI Specification}}

Το \en{OpenAPI Specification} είναι μία προδιαγραφή μοντελοποίησης για \en{RESTful} διεπαφές προγραμματισμού 
που χρησιμοποιείται για την περιγραφή, την ανάπτυξη, τη χρήση και την αναπαράστασή τους.

Η προδιαγραφή \en{OpenAPI} είναι ανεξάρτητη από τη γλώσσα προγραμματισμού και επιτρέπει 
τόσο στους προγραμματιστές να περιγράψουν εύκολα και κατανοητά μια διεπαφή, 
όσο και σε εφαρμογές να αλληλεπιδράσουν μαζί της.

Η μορφή που έχει είναι ένα αρχείο \en{JSON}
που περιλαμβάνει τα υποστηριζόμενα πεδία με τις τιμές τους.
Οι τιμές αυτές μπορεί να δηλώνουν είτε μία σταθερά είτε ένα σχήμα δεδομένων.

Οι τύποι δεδομένων που υποστηρίζονται είναι οι εξής:

\begin{itemize}
    \item Αριθμοί (\en{integer}, \en{long}, \en{float}, \en{double})
    \item Συμβολοσειρές (συμπεριλαμβανομένων \en{bytes} και \en{bits})
    \item Tύποι δεδομένων Αληθείας (\en{Boolean})
    \item Ειδικοί τύποι δεδομένων, όπως ημερομηνίες και κωδικοί πρόσβασης
\end{itemize}

Το αρχείο \en{JSON} της προδιαγραφής μπορεί να περιλαμβάνει τα ακόλουθα πεδία:

\begin{itemize}
    \item \underline{\en{openapi}}
    
    Υποχρεωτικό πεδίο, περιλαμβάνει την έκδοση της προδιαγραφής \en{OpenAPI} που χρησιμοποιείται.

    \item \underline{\en{info}}
    
    Υποχρεωτικό πεδίο, περιλαμβάνει απαραίτητες πληροφορίες για την διεπαφή.

    \item \underline{\en{servers}}
    
    Περιλαμβάνει τους διακομιστές με τους οποίους επικοινωνεί η διεπαφή.

    \item \underline{\en{paths}}
    
    Υποχρεωτικό πεδίο, περιλαμβάνει τα μονοπάτια των τελικών σημείων και των λειτουργιών τους.
    
    \item \underline{\en{components}}
    
    Περιλαμβάνει όλα τα δομικά στοιχεία της διεπαφής.
    
    \item \underline{\en{security}}
    
    Περιλαμβάνει περιγραφές των μηχανισμών ασφαλείας της διεπαφής.

    \item \underline{\en{tags}}
    
    Περιλαμβάνει ετικέτες με μεταδεδομένα (\en{metadata}) για την διεπαφή.
    
    \item \underline{\en{externalDocs}}
    
    Περιλαμβάνει εξωτερικά έγγραφα τεκμηρίωσης.
\end{itemize}

Το κομμάτι που παρουσιάζει ενδιαφέρον για την παρούσα εργασία είναι εκείνο των δομικών στοιχείων (\en{components}).
Όπως φαίνεται, χρησιμοποιώντας την προδιαγραφή \en{OpenAPI},
μπορεί κανείς να ορίσει μία προγραμματιστική διεπαφή τύπου \en{REST} ως ένα σύνολο από τα ακόλουθα συστατικά:

\begin{itemize}
    \item \underline{\en{schemas}}
    
    Περιγραφές για τα δεδομένα εισόδου ή εξόδου. Αυτές μπορεί να είναι μία κανονική έκφραση (\en{regex}) του περιεχομένου,
    μέγιστο και ελάχιστο μήκος ή πλήθος αντικειμένων κλπ.

    \item \underline{\en{responses}}
    
    Λίστα με τις αναμενόμενες απαντήσεις με βάση την κωδικό κατάστασης (\en{HTTP status code}).
    
    Κάθε απάντηση περιλαμβάνει υποχρεωτικά περιγραφική (\en{description}), 
    ενώ προαιρετικά πληροφορίες για τις κεφαλίδες και το περιεχόμενό της, 
    όπως το σχήμα των δεδομένων.

    \item \underline{\en{parameters}}
    
    Η προδιαγραφή \en{OpenAPI} ορίζει τέσσερις τύπους παραμέτρων:

    \begin{enumerate}
        \item \en{path} - μονοπατιού διεύθυνσης
        \item \en{query} - επιθέματος διεύθυνσης
        \item \en{header} - κεφαλίδας 
        \item \en{cookie} - δεδομένων \en{cookies}
    \end{enumerate}

    \item \underline{\en{examples}}
    
    Παραδείγματα επαλήθευσης με τιμές που στηρίζονται στο σχήμα δεδομένων του αντικειμένου.

    \item \underline{\en{requestBodies}}
    
    Αναπαράσταση των πιθανών σωμάτων μιας αίτησης, 
    με τις περιγραφές τους και πληροφορίες για το αν είναι προαιρετικές ή υποχρεωτικές
    και τι περιεχόμενο υποστηρίζουν.

    \item \underline{\en{headers}}
    
    Αναπαράσταση μιας κεφαλίδας,
    με δομή παρόμοια με αυτή των παραμέτρων.

    \item \underline{\en{securitySchemes}}
    
    Πληροφορίες για τις δυνατότητες ασφαλείας,
    με υποστηριζόμενες τις εξής:
    
    \begin{enumerate}
        \item Πιστοποίηση \en{HTTP}
        \item Κλειδί \en{API} (μέσω \en{header}, \en{query} ή \en{cookie})
        \item \en{OAuth2}
        \item \en{OpenID}
    \end{enumerate}

    \item \underline{\en{links}}
    
    Διευθύνσεις που αναπαριστούν πιθανές αιτήσεις της διεπαφής,
    με πληροφορίες για τις παραμέτρους τους,
    το σώμα τους και τον διακομιστή που επικοινωνούν.

    \item \underline{\en{callbacks}}
    
    Αναπαράσταση των πιθανών απαντήσεων της διεπαφής,
    με βάση τα χαρακτηριστικά των αιτήσεων.
\end{itemize}

Σε αντιπαραβολή με την παρούσα εργασία,
παρατηρούμε πως η μοντελοποίηση \en{RESTful API} που σχεδιάστηκε
υποστηρίζει περιορισμένους τύπους δεδομένων σε σχέση με το \en{OpenAPI}.
Για το ίδιο το \en{RESTful API} βλέπουμε πως εδώ υποστηρίζονται περισσότερα χαρακτηριστικά,
όπως πολλαπλοί διακομιστές και επιλογές ασφαλείας.
Όσο για τα δομικά στοιχεία της διεπαφής,
αξίζει να σημειωθούν η επιλογή για \en{cookies},
η πληθώρα μεθόδων και τύπων περιεχομένου (\en{content-type}) των αιτήσεων
και η μεγαλύτερη ποικιλία σε τύπους μεταβλητών, όπως για παράδειγμα ημερομηνίες και κωδικοί πρόσβασης.

\section{\en{Swagger Inspector}}
Το \en{Swagger} είναι ένα ανοικτό πλαίσιο εργαλείων που βοηθούν στον σχεδιασμό,
την ανάπτυξη, την τεκμηρίωση και την χρήστη προγραμματιστικών διεπαφών τύπου \en{REST}. 

Ένα από τα εργαλεία αυτά είναι το \en{Swagger Inspector} που επιτρέπει τον έλεγχο διεπαφών.
Ο έλεγχος πραγματοποιείται μέσα από την αλληλεπίδραση με τα τελικά σημεία μιας διεπαφής.

Το \en{Swagger Inspector} επιτρέπει την αποστολή αιτημάτων σε ένα τελικό σημείο 
χρησιμοποιώντας κάποιες βασικές \en{HTTP} μεθόδους.
Τα αιτήματα αυτά μπορεί να περιλαμβάνουν παραμέτρους, κεφαλίδες και σώμα.
Ακόμα υποστηρίζεται η διαπίστευση μέσω \en{username} και \en{password} σε \en{authentication header} και μέσω \en{JSON Web Token}.

Με την πραγματοποίηση της αίτησης,
εμφανίζεται η απάντηση που επιστρέφεται,
μαζί με τον κωδικό κατάστασης,
τον χρόνο που χρειάστηκε 
και τις κεφαλίδες που περιλαμβάνει.

Τέλος, υποστηρίζεται η εισαγωγή προδιαγραφής \en{OpenAPI}.
Αυτόματα δημιουργείται μια λίστα με τα υποστηριζόμενα τελικά σημεία 
και για καθένα από αυτά προκαθορισμένες αιτήσεις με τις αντίστοιχες μεθόδους.
Σε κάθε αίτηση περιλαμβάνονται οι παράμετροι που περιγράφονται στην προδιαγραφή
με τιμές που μπορεί να συμπληρώσει ο χρήστης.

Βλέπουμε πως το \en{Swagger Inspector} δεν παράγει αυτόματα σενάρια ελέγχου,
όμως σε αντίθεση με την παρούσα εργασία,
για κάθε χειροκίνητο σενάριο ελέγχου που εκτελείται, 
ενημερώνει τον χρήστη για τον χρόνο που χρειάστηκε.

\section{\en{Postman}}

Ένα άλλο πολύ δημοφιλές εργαλείο ελέγχου APIs είναι το \en{Postman} με 13 εκατομμύρια ενεργούς χρήστες και 500 χιλιάδες συνεργαζόμενες εταιρείες.

Πέρα από την πραγματοποίηση αιτημάτων σε τελικά σημεία και τη λήψη απαντήσεων,
έχει επιπλέον τη δυνατότητα συγγραφής και εκτέλεσης σεναρίων ελέγχου.
Αυτά έχουν τη μορφή τμημάτων κώδικα \en{JavaScript} 
και χρησιμοποιούν δυναμικές μεταβλητές 
για την αξιολόγηση των περιεχομένων της απάντησης  
και την ανταλλαγή δεδομένων μεταξύ αιτήσεων.

Για παράδειγμα,
ένας έλεγχος που είναι έγκυρος όταν η απάντηση έχει κωδικό κατάστασης 201 γράφεται σε \en{JavaScript} ως εξής:

\selectlanguage{english}
\begin{lstlisting}
pm.test("Status test", function () {
    pm.response.to.have.status(201);
});
\end{lstlisting}
\selectlanguage{greek}

Παρόμοια θα γράφαμε ένα σενάριο ελέγχου για το περιεχόμενο της απάντησης.
Αν θέλαμε να ελέγξουμε ότι η απάντηση είναι ένα αρχείο \en{JSON} και δεν δηλώνει κάποιο πρόβλημα,
θα συντάσσαμε τον ακόλουθο κώδικα:

\selectlanguage{english}
\begin{lstlisting}[deletekeywords={response}]
pm.test("response should be okay to process", function () {
    pm.response.to.not.be.error;
    pm.response.to.have.jsonBody("");
    pm.response.to.not.have.jsonBody("error");
});
\end{lstlisting}
\selectlanguage{greek}

Το \en{Postman} τέλος υποστηρίζει τη δημιουργία συλλογών και φακέλων με πλήθος σεναρίων ελέγχου.
Αυτά μπορεί να εκτελούνται είτε μετά από κάθε αίτηση,
είτε μετά από ένα σύνολο αιτήσεων,
ενώ μετά την εκτέλεσή τους παρουσιάζονται συνοπτικά πληροφορίες για κάθε αίτηση και κάθε σενάριο ελέγχου. 