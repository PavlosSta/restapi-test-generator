\chapter{Συμπεράσματα Και Μελλοντικές Επεκτάσεις}
\InitialCharacter{Β}ασική επιδίωξη της εργασίας είναι η αυτόματη παραγωγή σεναρίων ελέγχου για προγραμματιστικές διεπαφές τύπου \en{REST}.
Με τη δηλωτική γλώσσα ορισμού \en{RESTful API} μοντέλων που σχεδιάστηκε
μπορεί οποιοσδήποτε να περιγράψει μία διεπαφή προγραμματισμού.
Ο Κατασκευαστής \en{Groovy} διαβάζει τη γλώσσα αυτή
και παράγει \en{Java} κλάσεις που αναπαριστούν τη διεπαφή.
Έπειτα η Γεννήτρια Κώδικα δέχεται στην είσοδό της τις παραπάνω κλάσεις
και δημιουργεί σενάρια ελέγχου για το αρχικό \en{RESTful API} μοντέλο,
μαζί με τον απαραίτητο Πελάτη (\en{Client}) και σενάρια ελέγχου για τον παραγόμενο κώδικα. 
Όπως φαίνεται από τον σχεδιασμό και την υλοποίηση της εργασίας ,
καθώς και της επαλήθευσης της ορθής λειτουργίας της μέσω δύο διαδικτυακών \en{RESTful APIs},
με τη δηλωτική περιγραφή μιας διεπαφής προγραμματισμού μπορούμε εύκολα να παράγουμε σενάρια ελέγχου 
που καλύπτουν ένα μεγάλο φάσμα των δυνατοτήτων της και των χαρακτηριστικών της.

Στο πλαίσιο της εργασίας αποφασίστηκαν κάποιες παραδοχές ως προς την εμβάθυνση που θα γινόταν στο σύνολο των δεδομένων.
Ως εκ τούτου, υπάρχει περιθώριο μελλοντικών επεκτάσεων για την παραγωγή αναλυτικότερων και πιο συγκεκριμένων σεναρίων ελέγχου.
Για παράδειγμα, θα μπορούσε να υπάρξει υποστήριξη σε περισσότερους τύπους περιεχομένου (\en{content-type}) των αιτήσεων,
περισσότερων \en{HTTP} μεθόδων και μεγαλύτερη ποικιλία τύπων μεταβλητών.
Επιπλέον, χρήσιμη θα ήταν μία εξειδίκευση σε τελικά σημεία,
όπως για παράδειγμα αυτά της διαπίστευσης και αποσύνδεσης χρηστών,
που θα επέτρεπε πιο στοχευμένα σενάρια ελέγχου.
Τέλος, μία δυνατότητα επέκτασης της εργασίας αφορά την παροχή περισσότερων ειδών ελέγχου των προγραμματιστικών διεπαφών,
όπως για παράδειγμα έλεγχος της απόδοσής της.