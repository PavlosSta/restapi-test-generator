\chapter{Επαλήθευση Ορθής Λειτουργίας}
\InitialCharacter{Σ}το κεφάλαιο αυτό παρουσιάζονται οι εφαρμογές σε δύο διαδικτυακά \en{RESTful APIs} 
που επαληθεύουν την ορθή λειτουργία των εργαλείων που αναπτύχθηκαν στο πλαίσιο της παρούσας εργασίας.

\section{Παρατηρητήριο τιμών}
Η πρώτη προγραμματιστική διεπαφή τύπου \en{REST} που χρησιμοποιήθηκε για την επαλήθευση της εργασίας 
είναι τμήμα μιας εφαρμογής παρατηρητηρίου τιμών.
Ο κώδικας είναι διαθέσιμος στο διαδίκτυο σε \en{github repository}\footnote{\en{https://github.com/PavlosSta/SoftEng}}.

Τα χαρακτηριστικά της διεπαφής είναι τα ακόλουθα:

\begin{itemize}
    \item \textbf{Διεύθυνση}
    
    Ως διεύθυνση βάσης χρησιμοποιείται το '\en{https://localhost:8765/observatory/api}'.

    Οι πόροι του συστήματος δίνονται από τα τελικά σημεία στη μορφή \en{\{baseURL\}/\{path-to-resource\}}.

    Έτσι για παράδειγμα το τελικό σημείο για τα προϊόντα είναι το \en{\{baseURL\}/products},
    ενώ για το προϊόν με κωδικό 12 το \en{\{baseURL\}/products/12}.

    \item \textbf{Διαπίστευση} 
    
    Το τελικό σημείο διαπίστευσης χρηστών έχει το μονοπάτι '\en{\{baseURL\}/login}'.

    Η μοναδική μέθοδος που υποστηρίζει είναι η \en{POST}.

    Δέχεται ως παραμέτρους σώματος το \en{username} και το \en{password}, 
    ενώ σε περίπτωση επιτυχούς σύνδεσης επιστρέφει ένα \en{authentication token}.
    Αυτό θα πρέπει να περιλαμβάνεται ως κεφαλίδα με όνομα '\en{X-OBSERVATORY-AUTH}'
    σε όποια αίτηση απαιτείται πιστοποίηση.

    \item \textbf{Αποσύνδεση} 
    
    Τπ τελικό σημείο αποσύνδεσης χρηστών έχει το μονοπάτι '\en{\{baseURL\}/logout}'.

    Η μοναδική μέθοδος που υποστηρίζει είναι η \en{POST}.
    
    Θα πρέπει να έχει προηγηθεί διαπίστευση,
    η οποία θα έχει επιστρέφει ένα \en{authentication token}.
    Αυτό θα πρέπει να περιλαμβάνεται στην αίτηση ως κεφαλίδα ταυτοποίησης με όνομα '\en{X-OBSERVATORY-AUTH}'
    και αν είναι έγκυρο,
    η απάντηση θα είναι ένα \en{JSON} της μορφής '\en{\{ "message": "OK" \}}'.

    \item \textbf{Προϊόντα}
    
    Το τελικό σημείο των προϊόντων έχει το μονοπάτι '\en{\{baseURL\}/products}'.

    Σε αυτό υποστηρίζονται οι ακόλουθες μέθοδοι:

    \begin{enumerate}
        \item \underline{\textbf{\en{GET}}: Επιστρέφεται η λίστα των προϊόντων}
        
        Υποστηριζόμενες παράμετροι (ως μέρος του \en{URL query}):

        \selectlanguage{english}
        \begin{itemize}
            \item start: Integer, default 0
            \item count: Integer, default 20
            \item status: String, default ACTIVE
            \item sort: String, default id|DESC
        \end{itemize}
        \selectlanguage{greek}

        Για παράδειγμα:
        
        \en{GET} \emph{\en{\{baseURL\}/products?start=0\&count=100\&sort=id|ASC\&status=ALL}}

        Η απάντηση είναι ένα αρχείο \en{JSON} που περιλαμβάνει τις εξής παραμέτρους:

        \selectlanguage{english}
        \begin{itemize}
            \item start: Integer
            \item count: Integer
            \item total: Integer
            \item products: List<Product>
        \end{itemize}
        \selectlanguage{greek}

        Οι τρεις πρώτες παράμετροι αφορούν τη σελίδοποίηση των αποτελεσμάτων,
        ενώ η τελευταία περιλαμβάνει μία λίστα με τα προϊόντα της αντίστοιχης σελίδας.

        Κάθε προϊόν είναι ένα αρχείο \en{JSON} με τις εξής παραμέτρους:
        \selectlanguage{english}
        \begin{itemize}
            \item id: String
            \item name: String
            \item description: String
            \item category: String
            \item tags: List<String>
            \item withdrawn: Boolean
        \end{itemize}
        \selectlanguage{greek}


        \item \underline{\textbf{\en{POST}}: Δημιουργείται νέο προϊόν}
        
        Οι πληροφορίες που απαιτούνται για τη δημιουργία ενός προϊόντος 
        διατυπώνονται ως παράμετροι σώματος στην αίτηση.
        Αυτές είναι το \en{name}, το \en{description}, το \en{category} και τα \en{tags}.

        Προαιρετική παράμετρος είναι το \en{withdrawn}.
        
        Η απάντηση στην αίτηση δημιουργίας ενός προϊόντος 
        είναι η πλήρης κωδικοποίηση των δεδομένων του,
        δηλαδή ένα αρχείο \en{JSON} με τις αντίστοιχες πληροφορίες.
    \end{enumerate}

    Επιπλέον υποστηρίζει προσδιοριστικό (\en{attribute}) για το \en{id} των προϊόντων,
    με μονοπάτι της μορφής '\en{\{baseURL\}/products/\{id}\}'. 

    Με το \en{id} υποστηρίζονται οι ακόλουθες μέθοδοι:

    \begin{enumerate}
        \item \underline{\textbf{\en{GET}}: Επιστρέφονται οι πληροφορίες του προϊόντος}
        
        Για παράδειγμα, \en{GET} \emph{\en{\{baseURL\}/products/2}}.

        Το αποτέλεσμα της αίτησης είναι ένα αρχείο \en{JSON}
        με τα δεδομένα του προϊόντος με το αντίστοιχο \en{id}.

        Αυτά όπως και στην περίπτωση της μεθόδου \en{GET} χωρίς προσδιοριστικό \en{id}
        είναι τα ακόλουθα:

        \selectlanguage{english}
        \begin{itemize}
            \item id: String
            \item name: String
            \item description: String
            \item category: String
            \item tags: List<String>
            \item withdrawn: Boolean
        \end{itemize}
        \selectlanguage{greek}

        \item \underline{\textbf{\en{PUT}}: Ενημερώνονται οι πληροφορίες του προϊόντος}
        
        Η αίτηση πρέπει να περιλαμβάνει δεδομένα για όλα τα στοιχεία του προϊόντος, 
        τα οποία αντικαθιστούν τα προηγούμενα (\en{Full Update}).

        Τα δεδομένα αυτά περιλαμβάνονται στην αίτηση ως παράμετροι σώματος.

        Η απάντηση είναι η πλήρης κωδικοποίηση των δεδομένων του προϊόντος,
        δηλαδή ένα αρχείο \en{JSON} με τις αντίστοιχες πληροφορίες.

        \item \underline{\textbf{\en{PATCH}}: Ενημερώνονται μερικώς οι πληροφορίες του προϊόντος}
        
        Σε αντίθεση με τη μέθοδο \en{PUT},
        η \en{PATCH} επιτρέπει τη μερική επεξεργασία ενός προϊόντος (\en{Partial Update}).

        Τα δεδομένα για τα στοιχεία εκείνα που θέλουμε να ενημερώσουμε
        περιλαμβάνονται στην αίτηση ως παράμετροι σώματος.

        Η απάντηση είναι η πλήρης κωδικοποίηση των δεδομένων του προϊόντος,
        δηλαδή ένα αρχείο \en{JSON} με τις αντίστοιχες πληροφορίες.

        \item \underline{\textbf{\en{DELETE}}: Διαγράφεται το προϊόν}
        
        Το προϊόν με το αντίστοιχο \en{id} 
        αποκτά την τιμή \en{withdrawn=true} αν η αίτηση γίνεται από Εθελοντή,
        αλλιώς διαγράφεται από τη βάση δεδομένων αν γίνεται από Διαχειριστή.

        Το αποτέλεσμα είναι ένα αρχείο \en{JSON} της μορφής \en{\{ "message": "OK" \}}.
    \end{enumerate}
\end{itemize}
    
    Για λόγους συντομίας θα επικεντρωθούμε στα παραπάνω τελικά σημεία,
    αγνοώντας τα καταστήματα και τις τιμές,
    που έχουν όμοια λειτουργία με τα προϊόντα.

    Ας δούμε πώς μπορούμε να δηλώσουμε τις παραπάνω προδιαγραφές της διεπαφής 
    με σκοπό της παραγωγή σεναρίων ελέγχου.
    
    Καταρχάς, η διεύθυνσή του δίνεται ως η '\en{https://localhost:8765/observatory/api}',
    επομένως δηλώνουμε την θύρα του διακομιστή και τη διεύθυνση βάσης με τις εξής εντολές:

    \selectlanguage{english}
    \begin{lstlisting}[deletekeywords={api}]
    baseUrl '/observatory/api'
    serverPort = 8765
    \end{lstlisting}
    \selectlanguage{greek}

    Το πρώτο τελικό σημείο που έχουμε είναι αυτό της διαπίστευσης χρηστών.
    Παρατηρούμε πως υποστηρίζει μόνο τη μέθοδο \en{POST},
    για την οποία η αίτηση είναι τύπου \en{URL}
    με κωδικοποιημένες τις παραμέτρους σώματος \en{username} και \en{password}.
    Σε περίπτωση επιτυχούς σύνδεσης,
    έχουμε μία απάντηση με κατάσταση με κωδικό 201 και σώμα '\en{Created}'.

    Όλα τα παραπάνω γράφονται στον \en{Groovy Builder} με τον ακόλουθο τρόπο:

    \selectlanguage{english}
    \begin{lstlisting}
    endpoint('/login') {
    \end{lstlisting}
    \begin{lstlisting}[deletekeywords={endpoint}]
        label 'login endpoint'
        description 'the endpoint for user login'
        method('POST') {
            request('URL') {
                withBodyParameter('username', 'String')
                withBodyParameter('password', 'String')
            }
            response('JSON') {
                withStatus(201) {
                    body 'Created'
                }
            }
        }
    }
    \end{lstlisting}
    \selectlanguage{greek}

    Με όμοιο τρόπο περιγράφεται και το τελικό σημείο της αποσύνδεσης,
    το οποίο βέβαια αντί για \en{username} και \en{password} απαιτεί 
    μία κεφαλίδα με όνομα '\en{X-OBSERVATORY-AUTH}'.

    \selectlanguage{english}
    \begin{lstlisting}
    endpoint('/logout') {
    \end{lstlisting}
    \begin{lstlisting}[deletekeywords={endpoint}]
        label 'logout endpoint'
        description 'the endpoint for user logout'
        method('POST') {
            request('URL') {
                withHeader('X-OBSERVATORY-AUTH')
            }
            response('JSON') {
                withStatus(201) {
                    body 'Created'
                }
            }
        }
    }
    \end{lstlisting}
    \selectlanguage{greek}

    Σχετικά με το τελικό σημείο των προϊόντων,
    θα πρέπει να το δηλώσουμε μία φορά ως \en{endpoint} χωρίς προσδιοριστικό
    και άλλη μία ως \en{endpoint} με \en{id}.
    
    Για την περίπτωση χωρίς αναγνωριστικό,
    βλέπουμε ότι υποστηρίζονται οι μέθοδοι \en{GET} και \en{POST}.
    Αυτές δέχονται \en{URL} αιτήσεις 
    και οι απαντήσεις τους είναι σε μορφή \en{JSON}.

    Λαμβάνοντας υπόψη τις παραμέτρους και τις κεφαλίδες,
    προκύπτει ο ακόλουθος κώδικας για τον \en{Groovy Builder}:

    \selectlanguage{english}
    \begin{lstlisting}
    endpoint('/products') {
    \end{lstlisting}
    \begin{lstlisting}[deletekeywords={endpoint,description}]
        label 'products endpoint'
        method('GET') {
            request('URL') {
                withQueryParameter('start', 'Integer', 0)
                withQueryParameter('count', 'Integer', 20)
                withQueryParameter('status', 'String', 'ACTIVE')
                withQueryParameter('sort', 'String', 'id%7CDESC')
            }
            response('JSON') {
                withStatus(200) {
                    body 'OK'
                }
            }
        }
        method('POST') {
            request('URL') {
                withHeader('X-OBSERVATORY-AUTH')
                withBodyParameter('name', 'String')
                withBodyParameter('description', 'String')
                withBodyParameter('category', 'String')
                withBodyParameter('tags', 'String')
                withBodyParameter('withdrawn', 'boolean')
            }
            response('JSON') {
                withStatus(201) {
                    body 'OK'
                }
            }
        }
    }
    \end{lstlisting}
    \selectlanguage{greek}

    Στη συνέχεια έχουμε την περίπτωση ενός συγκεκριμένου προϊόντος με βάση το προσδιοριστικό \en{id} του.

    Δηλώνοντας τη μεταβλητή '\en{id}' μαζί με το μονοπάτι του τελικού σημείου
    και λαμβάνοντας υπόψη τις παραμέτρους και τις κεφαλίδες,
    προκύπτει ο ακόλουθος κώδικας για \en{Groovy Builder}:

    \selectlanguage{english}
    \begin{lstlisting}
    endpoint('/products', 'id') {
    \end{lstlisting}
    \begin{lstlisting}[deletekeywords={endpoint,description}]
        label 'products endpoint'
        method('GET') {
            request('URL') {}
            response('JSON') {
                withStatus(200) {
                    body 'OK'
                }
            }
        }
        method('PUT') {
            request('URL') {
                withHeader('X-OBSERVATORY-AUTH')
                withBodyParameter('name', 'String')
                withBodyParameter('description', 'String')
                withBodyParameter('category', 'String')
                withBodyParameter('tags', 'String')
                withBodyParameter('withdrawn', 'boolean')
            }
            response('JSON') {
                withStatus(201) {
                    body 'Created'
                }
            }
        }
        method('PATCH') {
            request('URL') {
                withHeader('X-OBSERVATORY-AUTH')
                withBodyParameter('name', 'String')
                withBodyParameter('description', 'String')
                withBodyParameter('category', 'String')
                withBodyParameter('tags', 'String')
                withBodyParameter('withdrawn', 'boolean')
            }
            response('JSON') {
                withStatus(201) {
                    body 'Created'
                }
            }
        }
        method('DELETE') {
            request('URL') {
                withHeader('X-OBSERVATORY-AUTH')
            }
            response('JSON') {
                withStatus(201) {
                    body 'Created'
                }
            }
        }
    }
    \end{lstlisting}
    \selectlanguage{greek}

    Εκτελώντας την εντολή \emph{\en{generate}} στο \en{Gradle},
    έχοντας πρώτα ορίσει και τις επιθυμητές τοποθεσίες,
    παράγονται τα εξής αρχεία:
    
    \begin{itemize}
        \item \underline{\en{REST API Client}}
        
        Το πρώτο και βασικό αρχείο που παράγεται είναι ο πελάτης (\en{client}) της διεπαφής.
        Αυτός πέρα από βοηθητικές διαδικασίες περιλαμβάνει για κάθε τελικό σημείο συναρτήσεις που υποστηρίζουν όλες τις μεθόδους του,
        μαζί με παραμέτρους και κεφαλίδες εφόσον αυτές παρέχονται.
        
        Για παράδειγμα, 
        η συνάρτηση που αντιστοιχεί στη μέθοδο \en{POST} του τελικού σημείου διαπίστευσης ('\en{login}') είναι η εξής:

        \selectlanguage{english}
        \begin{lstlisting}[language=java]
public Map<String, Object> post_to_login_with_headers_and_queryParams(String input, Map<String, String> headers, Map<String, List<String>> queryParams) {

    if(queryParams.isEmpty()) {
        return sendRequestAndParseResponseBodyAsUTF8Text(
                () -> newPostRequest(urlPrefix + "/login", "application/x-www-form-urlencoded", HttpRequest.BodyPublishers.ofString(input), headers),
                ClientHelper::parseJsonObject
        );
    }
    else {
        String queryParamString = queryParamsToString(queryParams);
        return sendRequestAndParseResponseBodyAsUTF8Text(
                () -> newPostRequest(urlPrefix + "/login" + queryParamString, "application/x-www-form-urlencoded", HttpRequest.BodyPublishers.ofString(input), headers),
                ClientHelper::parseJsonObject
        );
    }
}
\end{lstlisting}
\selectlanguage{greek}


Με παρόμοιο τρόπο δημιουργούνται οι υπόλοιπες μέθοδοι, 
ενώ καθεμιά, ανάλογα τον τύπο της, αξιοποιεί τις παρακάτω βοηθητικές διαδικασίες:

\selectlanguage{english}
\begin{lstlisting}[language=java]
private HttpRequest newPostRequest(String url, String contentType, HttpRequest.BodyPublisher bodyPublisher, Map<String, String> headers) {

    return newRequest("POST", url, contentType, bodyPublisher, headers);
}

private HttpRequest newGetRequest(String url, Map<String, String> headers) {

    return newRequest("GET", url, "application/x-www-form-urlencoded", HttpRequest.BodyPublishers.noBody(), headers);
}

private HttpRequest newPutRequest(String url, String contentType, HttpRequest.BodyPublisher bodyPublisher, Map<String, String> headers) {

    return newRequest("PUT", url, contentType, bodyPublisher, headers);
}

private HttpRequest newPatchRequest(String url, String contentType, HttpRequest.BodyPublisher bodyPublisher, Map<String, String> headers) {

    return newRequest("PATCH", url, contentType, bodyPublisher, headers);
}

private HttpRequest newDeleteRequest(String url, Map<String, String> headers) {

    return newRequest("DELETE", url, "application/x-www-form-urlencoded", HttpRequest.BodyPublishers.noBody(), headers);
}

\end{lstlisting}
\selectlanguage{greek}

Κάθε μέθοδος για να πραγματοποιήσει μια αίτηση χρησιμοποιεί την παρακάτω κοινή διαδικασία:

\selectlanguage{english}
\begin{lstlisting}[language=java]  
private HttpRequest newRequest(String method, String url, String contentType,
                                HttpRequest.BodyPublisher bodyPublisher, Map<String, String> headers) {

    HttpRequest.Builder builder = HttpRequest.newBuilder();

    builder.method(method, bodyPublisher)
            .header(CONTENT_TYPE_HEADER, contentType);

    for (Map.Entry<String,String> entry : headers.entrySet())
        builder.header(entry.getKey(), entry.getValue());

    return builder
            .uri(URI.create(url))
            .build();
}
\end{lstlisting}
\selectlanguage{greek}

Τέλος,
η πραγματοποίηση της \en{HTTP} αίτησης 
και η λήψη της απάντησης γίνονται από την ακόλουθη διαδικασία:

\selectlanguage{english}
\begin{lstlisting}[language=java]
private Map<String, Object> sendRequestAndParseResponseBodyAsUTF8Text(Supplier<HttpRequest> requestSupplier,
                                                                        Function<Reader, Map<String, Object>> bodyProcessor) {

    HttpRequest request = requestSupplier.get();

    Map<String, Object> bodyHeaders = new HashMap<>();

    try {
        System.out.println("Sending " + request.method() + " to " + request.uri());
        HttpResponse<InputStream> response = client.send(request, HttpResponse.BodyHandlers.ofInputStream());
        int statusCode = response.statusCode();
        if (statusCode == 200 || statusCode == 201) {
            try {
                if (bodyProcessor != null) {
                    bodyHeaders.put("headers", response.headers().map());
                    bodyHeaders.put("body", bodyProcessor.apply(new InputStreamReader(response.body(), StandardCharsets.UTF_8)));
                    return bodyHeaders;
                }
                else {
                    return null;
                }
            }
            catch(Exception e) {
                throw new ResponseProcessingException(e.getMessage(), e);
            }
        }
        else {
            throw new ServerResponseException(statusCode, ClientHelper.readContents(response.body()));
        }
    }
    catch(IOException | InterruptedException e) {
        throw new ConnectionException(e.getMessage(), e);
    }
}  
\end{lstlisting}
\selectlanguage{greek} 

Η παραπάνω διαδικασία υποστηρίζει αιτήσεις που έχουν μορφή \en{JSON}.
Με όμοιο τρόπο παράγονται για \en{String} και για \en{Integer}.
        
        \item \underline{Σενάρια ελέγχου για τον παραγόμενο κώδικα}
        
        Μαζί με τον παραπάνω \en{REST API Client},
        παράγονται και τα κατάλληλα σενάρια ελέγχου για αυτόν σε γλώσσα \en{Spock}.

        Για παράδειγμα, για το τελικό σημείο διαπίστευσης και τη μέθοδο \en{POST} που υποστηρίζει,
        δημιουργείται το παρακάτω σενάριο ελέγχου:
\selectlanguage{english}
\begin{lstlisting}[deletekeywords={api,body}]
def "POST to login with headers and queryParams"() {
    given:
    String requestBody = "username=bodyParamValue&password=bodyParamValue"
    ObjectMapper responseMapper = new ObjectMapper()
    JsonNode resultJSON = responseMapper.readTree("{\"value\":\"ok\"}")
    wms.givenThat(
            post(urlMatching("/observatory/api/login\\\\?.*"))
            .withRequestBody(containing(requestBody))
            .willReturn(aResponse()
                    .withStatus(201)
                    .withJsonBody(resultJSON)
            )
    )
    Map<String, String> headers = new HashMap<>()
    Map<String, List<String>> queryParams = new HashMap<>()

    when:
    Map<String, Object> result = caller.post_to_login_with_headers_and_queryParams(requestBody, headers, queryParams)

    then:
    result.get("body").toString().matches("[\\{\\[].*[\\}\\]]")
}
\end{lstlisting}
\selectlanguage{greek} 

Βλέπουμε ότι προσδιορίζεται ένας εικονικός διακομιστής (\en{Mock Server}),
που λαμβάνει τις αιτήσεις του \en{REST API Client}
με βάση μία κανονική έκφραση (\en{regular expression}) που ελέγχει τη διεύθυνση.

Παράλληλα δημιουργούνται δοκιμαστικές αιτήσεις με βάση τις ιδιότητες της μεθόδου και του τελικού σημείου.
Στο παράδειγμά μας οι παράμετροι σώματος '\en{username}' και '\en{password}' παίρνουν τυχαίες τιμές 
και αποκτούν την κωδικοποιημένη μορφή \en{URL}.

Ο εικονικός διακομιστής αναμένει αίτηση που περιλαμβάνει το περιεχόμενο που στέλνουμε
και επιστρέφει μία απάντηση με κωδικό κατάστασης 201 και ένα τυχαίο αρχείο \en{JSON}.

Ο τελευταίος έλεγχος αφορά την απάντηση που δέχεται ο \en{REST API Client} από τον εικονικό διακομιστή
και μέσω κανονικής έκφρασης ελέγχει αν είναι μορφής \en{JSON}. 


        \item \underline{Σενάρια ελέγχου για το \en{RESTful API}}
        
        Το τελευταίο αρχείο που παράγεται περιλαμβάνει σενάρια ελέγχου για τον '\en{RESTful API}',
        αυτή τη φορά θεωρώντας πως έχουμε πραγματικό διακομιστή αντί για εικονικό.
\selectlanguage{english}
\begin{lstlisting}[deletekeywords={body}]
def "POST to login with headers and queryParams"() {

    given:
    String requestBody = "username=bodyParamValue&password=bodyParamValue"
    Map<String, String> headers = new HashMap<>()
    Map<String, List<Object>> queryParams = new HashMap<>()

    when:
    Map<String, Object> result = caller.post_to_login_with_headers_and_queryParams(requestBody, headers, queryParams)
    Map<String, Object> resultHeaders = result.get("headers") as Map<String, Object>

    then:
    println("Body:")
    println(result.get("body"))
    println("Headers:")
    println(resultHeaders)
    result.toString().matches("[\\{\\[].*[\\}\\]]")
}
\end{lstlisting}
\selectlanguage{greek} 

Όπως βλέπουμε,
παρότι λείπει ο εικονικός διακομιστής,
δημιουργούμε μία αίτηση με βάση τις παραμέτρους και τις κεφαλίδες που υποστηρίζονται.

Στα σενάρια ελέγχου για το \en{RESTful API} έχουμε επιπλέον την εκτύπωση του σώματος και των κεφαλίδων της απάντησης.
Αυτό είναι χρήσιμες σε περιπτώσεις όπως αυτή του τελικού σημείου διαπίστευσης,
μιας και ο χρήστης μπορεί εύκολα να αντικαταστήσει τις προκαθορισμένες τιμές με ένα λειτουργικό ζεύγος 
\en{username} και \en{password} και να πάρει το \en{authentication token} είτε αυτό επιστρέφεται 
μέσω του σώματος της απάντησης,
είτε μέσω κάποιας κεφαλίδας της.

Εκτελώντας το παραπάνω σενάριο ελέγχου,
έχουμε την ακόλουθη έξοδο στην κονσόλα:

\selectlanguage{english}
\begin{lstlisting}[deletekeywords={api}]
Sending POST to https://localhost:8765/observatory/api/login
Body:
[token:$2a$10$hwOh6PJZeV1Nl9HKmOwsdO7RLWPxwEgojuYINKvDWwAYOkWvK11Uu]
Headers:
[content-length:[75], content-type:[application/json], date:[Mon, 11 Jan 2021 10:13:41 GMT], vary:[Origin, Access-Control-Request-Method, Access-Control-Request-Headers]]
\end{lstlisting}
\selectlanguage{greek} 

Επιπλέον παράγονται έλεγχοι για κάθε παράμετρο και κεφαλίδα που είναι υποχρεωτικές.
Στην περίπτωσή μας και οι δύο παράμετροι σώματος της αίτησης (\en{username} και \en{password}) είναι υποχρεωτικές.
Επομένως το ακόλουθο σενάριο ελέγχου πραγματοποιεί μία αίτηση χωρίς \en{username}
και θεωρείται έγκυρο αν ο διακομιστής εγείρει εξαίρεση.

\selectlanguage{english}
\begin{lstlisting}
def "POST to login without mandatory bodyParam: username"() {

    given:
    String requestBody = "password=bodyParamValue"
    Map<String, String> headers = new HashMap<>()
    Map<String, List<Object>> queryParams = new HashMap<>()

    when:
    caller.post_to_login_with_headers_and_queryParams(requestBody, headers, queryParams)

    then:
    thrown RuntimeException
}
\end{lstlisting}
\selectlanguage{greek} 
    \end{itemize}

\section{Ψηφιακό μητρώο δικτυακών υποδομών}
Η δεύτερη προγραμματιστική διεπαφή τύπου \en{REST} που χρησιμοποιήθηκε για την επαλήθευση της εργασίας
είναι τμήμα ενός ψηφιακού μητρώου για της δικτυακές υποδομές μιας χώρας.
Σκοπός του είναι η καταγραφή και η παρακολούθηση της κατάστασης
του συνόλου των διαθέσιμων δικτυακών υποδομών της χώρας.

Πιο συγκεκριμένα, η διεπαφή έχει τα εξής χαρακτηριστικά:

\begin{itemize}
    \item \textbf{Διεύθυνση}
    
    Ως διεύθυνση βάσης χρησιμοποιείται το '\en{https://localhost:44349/api}'.

    Οι πόροι του συστήματος δίνονται από τα τελικά σημεία στη μορφή \en{\{baseURL\}/\{path-to-resource\}}.

    Έτσι το τελικό σημείο των παρόχων είναι το \en{\{baseURL\}/NationalNetwork},
    ενώ για τον πάροχο με κωδικό 3 το \en{\{baseURL\}/NationalNetwork/3}.

    \item \textbf{Διαπίστευση} 
    
    Το τελικό σημείο διαπίστευσης χρηστών έχει το μονοπάτι '\en{\{baseURL\}/login}'.

    Η μοναδική μέθοδος που υποστηρίζει είναι η \en{POST}.

    Υπάρχουν δύο ειδών χρήστες στο σύστημα,
    οι Διαχειριστές που έχουν δικαίωμα επεξεργασίας όλων των δεδομένων
    και οι Πάροχοι που έχουν δικαίωμα επεξεργασίας μόνο των πληροφοριών των συνδέσεων.

    Το τελικό σημείο της διαπίστευσης δέχεται ένα αρχείο \en{JSON} με το \en{username} και το \en{password}.
    Σε περίπτωση επιτυχούς σύνδεσης επιστρέφει αρχείο \en{JSON} με το \en{id} του χρήστη,
    το \en{username} του, τον ρόλο του και ένα \en{Bearer token}.
    Αυτό θα πρέπει να περιλαμβάνεται ως κεφαλίδα ταυτοποίησης με όνομα '\en{Authorization}'
    σε όποια αίτηση απαιτείται πιστοποίηση.

    Η διάρκεια του \en{JSON Web Token} είναι 7 ημέρες.

    \item \textbf{Πάροχοι}
    
    Το τελικό σημείο των παρόχων έχει το μονοπάτι '\en{\{baseURL\}/NationalNetwork}'.

    Σε αυτό υποστηρίζονται οι ακόλουθες μέθοδοι:

    \begin{enumerate}
        \item \underline{\textbf{\en{GET}}: Επιστρέφεται η λίστα των παρόχων}
        
        Δικαίωμα προβολής των πληροφοριών των παρόχων έχει οποιοσδήποτε χρήστης,
        χωρίς ανάγκη πιστοποίησης.

        Η απάντηση είναι ένα αρχείο \en{JSON} που περιλαμβάνει μία λίστα με τα στοιχεία των παρόχων,
        της οποίας κάθε καταχώρηση περιλαμβάνει τις εξής παραμέτρους:

        \selectlanguage{english}
        \begin{itemize}
            \item id: Integer
            \item name: String
            \item location: String
            \item created: String
            \item email: String
        \end{itemize}
        \selectlanguage{greek}

        \item \underline{\textbf{\en{POST}}: Δημιουργείται νέα καταχώρηση παρόχου}
        
        Οι πληροφορίες που απαιτούνται για τη δημιουργία ενός παρόχου 
        διατυπώνονται ως παράμετροι σώματος στην αίτηση.
        Αυτές είναι το \en{name}, το \en{location} και το \en{email}.
        
        Η απάντηση στην αίτηση δημιουργίας ενός προϊόντος 
        είναι η πλήρης κωδικοποίηση των δεδομένων του,
        δηλαδή ένα αρχείο \en{JSON} με τις αντίστοιχες πληροφορίες.

    \end{enumerate}

    Επιπλέον υποστηρίζει προσδιοριστικό (\en{attribute}) για το \en{id} των παρόχων,
    με μονοπάτι της μορφής '\en{\{baseURL\}/NationalNetwork/\{id}\}'. 

    Με το \en{id} υποστηρίζονται οι ακόλουθες μέθοδοι:

    \begin{enumerate}
        \item \underline{\textbf{\en{GET}}: Επιστρέφονται οι πληροφορίες του παρόχου}
        
        Για παράδειγμα, \en{GET} \emph{\en{\{baseURL\}/products/2}}.

        Το αποτέλεσμα της αίτησης είναι ένα αρχείο \en{JSON}
        με τα δεδομένα του παρόχου με το αντίστοιχο \en{id}.

        Αυτά όπως και στην περίπτωση της μεθόδου \en{GET} χωρίς προσδιοριστικό \en{id}
        είναι τα ακόλουθα:

        \selectlanguage{english}
        \begin{itemize}
            \item id: Integer
            \item name: String
            \item location: String
            \item created: String
            \item email: String
        \end{itemize}
        \selectlanguage{greek}

        \item \underline{\textbf{\en{PATCH}}: Ενημερώνονται μερικώς οι πληροφορίες του παρόχου}
        
        Η μέθοδος \en{PATCH} επιτρέπει τη μερική επεξεργασία ενός παρόχου (\en{Partial Update}).

        Τα δεδομένα για τα στοιχεία εκείνα που θέλουμε να ενημερώσουμε
        περιλαμβάνονται στην αίτηση ως παράμετροι σώματος.

        Η απάντηση είναι η πλήρης κωδικοποίηση των δεδομένων του παρόχου,
        δηλαδή ένα αρχείο \en{JSON} με τις αντίστοιχες πληροφορίες.

        \item \underline{\textbf{\en{DELETE}}: Διαγράφεται ο πάροχος}
        
        Ο πάροχος με το αντίστοιχο \en{id} διαγράφεται από τη βάση δεδομένων,
        εφόσον η αίτηση γίνεται από χρήστη με ρόλο Διαχειριστή.

        Το αποτέλεσμα είναι ένα αρχείο \en{String} της μορφής \en{"success at deletion"}.
    \end{enumerate}
\end{itemize}

Για λόγους συντομίας θα ασχοληθούμε μόνο με το τελικό σημείο των παρόχων.
Όμοια με αυτό λειτουργούν τα τελικά σημεία των συνδέσεων και των χρηστών,
ενώ η διαπίστευση δε διαφέρει σημαντικά από αυτήν του παρατηρητηρίου τιμών του πρώτου παραδείγματος.

Η διεύθυνση της διεπαφής δίνεται ως η '\en{https://localhost:44349/api}',
επομένως δηλώνουμε την θύρα του διακομιστή και τη διεύθυνση βάσης με τις εξής εντολές:

\selectlanguage{english}
\begin{lstlisting}[deletekeywords={api}]
baseUrl '/api'
serverPort = 44349
\end{lstlisting}
\selectlanguage{greek}

Όπως και πριν,
έτσι και τώρα θα δηλώσουμε μία φορά το τελικό σημείο χωρίς προσδιοριστικό 
και μία με το \en{id}.

Μία βασική διαφορά των δύο διεπαφών είναι ότι το ψηφιακό μητρώο δικτυακών υποδομών
δέχεται αιτήσεις με τύπο περιεχομένου \en{JSON} αντί για \en{URL}.
Αυτό όμως δεν είναι πρόβλημα ούτε απαιτεί κάποια σημαντική αλλαγή πέρα από τη δήλωση '\en{JSON}' 
στην αίτηση κάθε μεθόδου.

Έτσι, για παράδειγμα, 
για να περιγράψουμε στον \en{Groovy Builder} το τελικό σημείο των παρόχων με και χωρίς προσδιοριστικό,
γράφουμε τα εξής:

\selectlanguage{english}
\begin{lstlisting}
endpoint('/NationalNetwork') {
\end{lstlisting}
\begin{lstlisting}[deletekeywords={endpoint}]
    label 'providers networks'
    description 'an endpoint for providers'
    method('GET') {
        request('JSON') {}
        response('JSON') {
            withStatus(200) {
                body 'OK'
            }
        }
    }
    method('POST') {
        request('JSON') {
            withHeader('Authorization')
            withBodyParameter('name', 'String')
            withBodyParameter('location', 'String')
            withBodyParameter('email', 'String')
        }
        response('JSON') {
                withStatus(201) {
                    body 'Created'
                }
        }
    }
}
\end{lstlisting}
\begin{lstlisting}
endpoint('/NationalNetwork', 'id') {
\end{lstlisting}
\begin{lstlisting}[deletekeywords={endpoint}]
    label 'providers networks with id'
    description 'an endpoint for providers by id'
    method('PATCH') {
        request('JSON') {
            withHeader('Authorization')
            withBodyParameter('name', 'String')
            withBodyParameter('location', 'String')
            withBodyParameter('email', 'String')
        }
        response('JSON') {
            withStatus(200) {
                body 'OK'
            }
        }
    }
    method('DELETE') {
        request('JSON') {
            withHeader('Authorization')
            withBodyParameter('name', 'String')
            withBodyParameter('location', 'String')
            withBodyParameter('email', 'String')
        }
        response('String') {
            withStatus(201) {
                body 'Created'
            }
        }
    }
}
\end{lstlisting}
\selectlanguage{greek}

Με την εκτέλεση της εντολής \emph{\en{generate}} στον \en{Gradle},
πάραγονται όπως και πριν τα παρακάτω τρία αρχεία:

\begin{itemize}
    \item \underline{\en{REST API Client}}
    
    Βλέπουμε ότι στον πελάτη που δημιουργείται, 
    για κάθε μέθοδο της διεπαφής που δέχεται αιτήσεις τύπου \en{JSON},
    αυτόματα αλλάζει το όρισμα του τύπου δεδομένων σε '\en{application/json}'.    

    \selectlanguage{english}
    \begin{lstlisting}[language=java]
    public Map<String, Object> post_to_NationalNetwork_with_headers_and_queryParams(String input, Map<String, String> headers, Map<String, List<String>> queryParams) {

        if(queryParams.isEmpty()) {
            return sendRequestAndParseResponseBodyAsUTF8Text(
                    () -> newPostRequest(urlPrefix + "/NationalNetwork", "application/json", HttpRequest.BodyPublishers.ofString(input), headers),
                    ClientHelper::parseJsonObject
            );
        }
        else {
        String queryParamString = queryParamsToString(queryParams);

            return sendRequestAndParseResponseBodyAsUTF8Text(
                    () -> newPostRequest(urlPrefix + "/NationalNetwork" + queryParamString, "application/json", HttpRequest.BodyPublishers.ofString(input), headers),
                    ClientHelper::parseJsonObject
            );
        }
    }
    \end{lstlisting}
    \selectlanguage{greek}

    \item \underline{Σενάρια ελέγχου για τον παραγόμενο κώδικα}

    Στα σενάρια ελέγχου που παράγονται και αφορούν τον κώδικα του \en{REST API Client}
    αυτή τη φορά έχουμε τελικό σημείο διαπίστευσης που δέχεται τα δεδομένα σε μορφή \en{JSON}.
    Δηλώνοντάς το αυτό στον \en{Groovy Builder},
    το αντίστοιχο σενάριο ελέγχου είναι το εξής:

    \selectlanguage{english}
    \begin{lstlisting}[deletekeywords={api,body}]
    def "POST to users_authenticate with headers and queryParams"() {

        given:
        String requestBody = new JSONObject()
            .put("username", "bodyParamValue")
            .put("password", "bodyParamValue")
            .toString();
        ObjectMapper responseMapper = new ObjectMapper()
        JsonNode resultJSON = responseMapper.readTree("{\"value\":\"ok\"}")
        wms.givenThat(
                post(urlMatching("/api/users/authenticate\\\\?.*"))
                .withRequestBody(containing(requestBody))
                .willReturn(aResponse()
                        .withStatus(201)
                        .withJsonBody(resultJSON)
                )
        )
        Map<String, String> headers = new HashMap<>()
        Map<String, List<String>> queryParams = new HashMap<>()
        
        when:
        Map<String, Object> result = caller.post_to_users_authenticate_with_headers_and_queryParams(requestBody, headers, queryParams)
        
        then:
        result.get("body").toString().matches("[\\{\\[].*[\\}\\]]")
    }
\end{lstlisting}
\selectlanguage{greek}

    \item \underline{Σενάρια ελέγχου για το \en{RESTful API}}
    
    Όμοια με παραπάνω,
    το σενάριο ελέγχου για την διαπίστευση των χρηστών προσαρμόζεται στα δεδομένα εισόδου 
    και γίνεται το ακόλουθο:

    \selectlanguage{english}
    \begin{lstlisting}[deletekeywords={body}]
    def "POST to users_authenticate with headers and queryParams"() {

        given:
        String requestBody = new JSONObject()
            .put("username", "bodyParamValue")
            .put("password", "bodyParamValue")
            .toString();
        Map<String, String> headers = new HashMap<>()
        Map<String, List<Object>> queryParams = new HashMap<>()

        when:
        Map<String, Object> result = caller.post_to_users_authenticate_with_headers_and_queryParams(requestBody, headers, queryParams)
        Map<String, Object> resultHeaders = result.get("headers") as Map<String, Object>

        then:
        println("Body:")
        println(result.get("body"))
        println("Headers:")
        println(resultHeaders)

        result.toString().matches("[\\{\\[].*[\\}\\]]")
    }
\end{lstlisting}
\selectlanguage{greek}

    Αυτή τη φορά μπορούμε να τροποποιήσουμε το σενάριο ελέγχου 
    αλλάζοντας τις τυχαίες τιμές των παραμέτρων \en{username} και \en{password} στο \en{JSON} αρχείο αίτησης
    με ένα ζεύγος έγκυρων δεδομένων.
    
    Το \en{authentication token} που θα τυπωθεί στην κονσόλα μπορούμε να το αντιγράψουμε 
    στην τιμή της αντίστοιχης κεφαλίδας πιστοποίησης των σεναρίων ελέγχου που την απαιτούν.
    
    Έτσι, για παράδειγμα, μπορούμε να αντικαταστήσουμε με αυτό το '\en{headerValue}' στο σενάριο ελέγχου της δημιουργίας 
    νέου παρόχου μέσω μεθόθου \en{POST}, το οποίο παράγεται αυτόματα ως εξής:

    \selectlanguage{english}
    \begin{lstlisting}[deletekeywords={body}]
def "POST to NationalNetwork with headers and queryParams"() {

    given:
    String requestBody = new JSONObject()
        .put("name", "bodyParamValue")
        .put("location", "bodyParamValue")
        .put("email", "bodyParamValue")
        .toString();
    Map<String, String> headers = new HashMap<>()
    headers.put("Authorization", 'headerValue')
    Map<String, List<Object>> queryParams = new HashMap<>()

    when:
    Map<String, Object> result = caller.post_to_NationalNetwork_with_headers_and_queryParams(requestBody, headers, queryParams)
    Map<String, Object> resultHeaders = result.get("headers") as Map<String, Object>

    then:
    println("Body:")
    println(result.get("body"))
    println("Headers:")
    println(resultHeaders)

    result.toString().matches("[\\{\\[].*[\\}\\]]")
}
\end{lstlisting}
\selectlanguage{greek}
    
\end{itemize}